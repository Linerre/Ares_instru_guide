\chapter{Introduction}
\label{ch:intro}
\section{What's in a name?}
\label{sec:}
\ares stands for {\imp A}utomated {\imp Res}erves. It is a single access point for instructors and students for all course-related materials. 

\ares allows instructors to put materials on reserve for their courses, either in print or electronically, That is then available to students enrolled in the courses. It can even be integrated with NYU Classes. 

This Guide will navigate instructors through \ares web interface. For the impatient, you may directly go to~\autoref{ch:add} on adding reserve items.

\section{Key features}
\ares comes in handy in terms of managing your course reserves in collaboration with library staff.
It can:
\begin{itemize}
    \item Reduce reliance on email communication: Once submitted, the processing status of a request can be tracked within \ares system;
    \item Access all reserve materials (physical and electronic) at one central and secure site;
    \item Provide links to e-journal articles and e-books that are stable and accessible from on- and -off campus;
    \item Integrate with NYU Classes;
    \item Clone items from previous or current courses;
    \item Add proxy users who can request reserve items on behalf of an instructor;
    \item Customize interface when viewing course reserve readings.
\end{itemize}

\vspace*{3ex}
\begin{table}[h]
    \centering
    \begin{notebox}
    Please be advised, {\imp required textbooks} that each student needs to purchase will be made available for your class through \ares by library staff. Therefore, instructors normally {\imp DO NOT} need to submit request for required textbooks. 
    \tcblower
    For {\imp reference materials}, {\imp recommended readings (books, book chapters, journal articles etc.)} and {\imp videos}, please go ahead and submit them through \ares following the instructions. 
    \end{notebox}
    \label{note: required texts}
\end{table}
